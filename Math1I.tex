%%%%%%%%%%%%%%%%%%%%%%%%%%%%%%%%%%%%%%%%%%%%%%%%%%%%%%%%%%%%%%%%%%%%%%%%
%
% Zusammenfassung Mathematische Grundlagen der Informatik 1 HS2012
%
%%%%%%%%%%%%%%%%%%%%%%%%%%%%%%%%%%%%%%%%%%%%%%%%%%%%%%%%%%%%%%%%%%%%%%%%

%%%%%%%%%%%%%%%%%%%%%%%%%%%%%%%%%%%%%%%%%%%%%%%%%%%%%%%%%%%%%%%%%%%%%%%%
% Variabeln
\newcommand{\SUBJECT}{Zusammenfassung Math1I HS2012}
\newcommand{\TITLE}{Mathematische Grundlagen der Informatik 1}
\newcommand{\SEMESTER}{1. Semester (HS 2012)}
\newcommand{\AUTHOR}{Emanuel Duss}
\newcommand{\EMAIL}{emanuel.duss@gmail.com}
\newcommand{\KEYWORDS}{Zusammenfassung, Informatik, HSR}

\newcommand{\highlight}[1]{\textbf{#1}}

%%%%%%%%%%%%%%%%%%%%%%%%%%%%%%%%%%%%%%%%%%%%%%%%%%%%%%%%%%%%%%%%%%%%%%%%
% Header einbinden
%%%%%%%%%%%%%%%%%%%%%%%%%%%%%%%%%%%%%%%%%%%%%%%%%%%%%%%%%%%%%%%%%%%%%%%%
%
% header.tex
%
%%%%%%%%%%%%%%%%%%%%%%%%%%%%%%%%%%%%%%%%%%%%%%%%%%%%%%%%%%%%%%%%%%%%%%%%

%%%%%%%%%%%%%%%%%%%%%%%%%%%%%%%%%%%%%%%%%%%%%%%%%%%%%%%%%%%%%%%%%%%%%%%%
% Dokument
\documentclass[12pt,oneside,a4paper,parskip]{scrartcl}
\usepackage[inner=2.5cm,outer=2.5cm,top=2.5cm,bottom=2.5cm,includeheadfoot]{geometry}  % Seitenrand

% Palatino Schriftart
\usepackage[sc]{mathpazo}    % Palatino Schriftart
\renewcommand\sfdefault{ppl}

% Sprache, Kodierung
\usepackage{ngerman}
\usepackage[T1]{fontenc}
\usepackage[utf8]{inputenc}

% Weitere Pakete
\usepackage{graphicx}
\usepackage{color}
\usepackage[table]{xcolor} % http://ctan.org/pkg/xcolor
\usepackage{fancybox}
\usepackage[colorlinks=true, linkcolor=blue, urlcolor=blue, citecolor=blue]{hyperref}
\usepackage{amsmath}
\usepackage{amssymb}
\usepackage{comment}

% Jede Überschrift 1 auf neuer Seite
\let\stdsection\section
\renewcommand\section{\newpage\stdsection}

%%%%%%%%%%%%%%%%%%%%%%%%%%%%%%%%%%%%%%%%%%%%%%%%%%%%%%%%%%%%%%%%%%%%%%%%
% Angaben zum Dokument
\subject{\SUBJECT}
\title{\TITLE}
\author{\AUTHOR \\ \EMAIL}

%%%%%%%%%%%%%%%%%%%%%%%%%%%%%%%%%%%%%%%%%%%%%%%%%%%%%%%%%%%%%%%%%%%%%%%%
% PDF-Metainformationen
\hypersetup{
  pdftitle    = {\TITLE},
  pdfsubject  = {\SUBJECT},
  pdfauthor   = {\AUTHOR, \EMAIL},
  pdfkeywords = {\KEYWORDS} ,
  pdfcreator  = {pdflatex},
  pdfproducer = {LaTeX with hyperref}
}

%%%%%%%%%%%%%%%%%%%%%%%%%%%%%%%%%%%%%%%%%%%%%%%%%%%%%%%%%%%%%%%%%%%%%%%%
% Kopf- und Fusszeile
\usepackage{fancyhdr}
\pagestyle{fancy}
\fancyhf{}
\renewcommand{\headrulewidth}{0.5pt}
\fancyhead[L]{\nouppercase{\TITLE}}
\fancyhead[R]{\nouppercase{\leftmark}}
\renewcommand{\footrulewidth}{0.5pt}
\fancyfoot[L]{\AUTHOR}
\fancyfoot[R]{\thepage}

%%%%%%%%%%%%%%%%%%%%%%%%%%%%%%%%%%%%%%%%%%%%%%%%%%%%%%%%%%%%%%%%%%%%%%%%
% Dokument start
\begin{document}

% Titelseite
\begin{titlepage}
  \maketitle
  \thispagestyle{empty} % Don't start page numbers on this page
  \vfill
  \begin{figure}[!htbp]
    \centering
    \includegraphics[width=300px]{hsr_logo.pdf}
  \end{figure}
\end{titlepage}

% Inhaltsverzeichnis
\setcounter{page}{1}
\tableofcontents

% CC BY-SA Lizenz
\small{
\vfill
Diese Zusammenfassung basiert auf der Vorlesung und auf dem Skript von \emph{\TITLE} der
HSR vom Herbstsemester 2012. \\ \\
\begin{center}
\includegraphics[width=130px]{cc.pdf}
\end{center}
CC BY-SA by Emanuel Duss
(\href{mailto:emanuel.duss@gmail.com}{\tt emanuel.duss@gmail.com})
  % \href{http://emanuelduss.ch}{\tt http://emanuelduss.ch}
% \begin{center}
% 
% 
% Dieses Dokument steht unter der \\
% \emph{Creative Commons \\ Namensnennung - Weitergabe unter gleichen Bedingungen 3.0 Schweiz Lizenz} \\
% 
 % \begin{figure}[!htbp]
   % \centering
 % \end{figure}
% 
% \href{http://creativecommons.org/licenses/by-sa/3.0/ch/}{\tt http://creativecommons.org/licenses/by-sa/3.0/ch/}
% \end{center}
}
\newpage

% EOF

%%%%%%%%%%%%%%%%%%%%%%%%%%%%%%%%%%%%%%%%%%%%%%%%%%%%%%%%%%%%%%%%%%%%%%%%

%%%%%%%%%%%%%%%%%%%%%%%%%%%%%%%%%%%%%%%%%%%%%%%%%%%%%%%%%%%%%%%%%%%%%%%%
% Inhalt Start
%%%%%%%%%%%%%%%%%%%%%%%%%%%%%%%%%%%%%%%%%%%%%%%%%%%%%%%%%%%%%%%%%%%%%%%%

%%%%%%%%%%%%%%%%%%%%%%%%%%%%%%%%%%%%%%%%%%%%%%%%%%%%%%%%%%%%%%%%%%%%%%%%
\section{Aussagenlogik}
\subsection{Aussage}
Eine Aussage ist ein Satz, welcher entweder falsch oder wahr ist.

\subsection{Junktoren}
\begin{itemize}
  \item $\neg$ Negation (nicht): Wahr, wenn die Aussage falsch ist.
  \item $\wedge$ Konjunktion (und): Wahr, wenn beide Aussagen wahr sind.
  \item $\vee$ Disjunktion (oder): Wahr wenn eine der beiden Aussagen wahr ist.
  \item $\Rightarrow$ Implikation (wenn \dots dann):
    \begin{itemize}
      \item Beispiel: $A \Rightarrow B$
      \item Wenn Aussage $A$ gilt, dann gilt auch Aussage $B$.
      \item Das heisst: Nur falsch, wenn $A$ wahr und $B$ falsch ist.
    \end{itemize}
  \item $\Leftrightarrow$ Äquivalenz (genau dann, wenn \dots)
    \begin{itemize}
      \item Beispiel: $A \Leftrightarrow B$
      \item Wahr, wenn $A$ und $B$ den gleichen Wahrheitswert besitzen.
    \end{itemize}
  \item $\uparrow$ NAND (Zusammengesetzt aus NOT (nicht, $\neg$) und AND (und $\wedge$))
    \begin{itemize}
      \item Alle Junktoren können durch das NAND ausgedrückt werden
      \item Beispiel: $A \uparrow B \Leftrightarrow \neg (A \wedge B)$ sowie $A \uparrow A \Leftrightarrow \neg A$ \\
       sowie $A \wedge B \Leftrightarrow (A \uparrow B) \uparrow (A \uparrow B)$ sowie
         $A \vee B \Leftrightarrow (A \uparrow A) \uparrow (B \uparrow B)$
    \end{itemize}
\end{itemize}

\subsection{Wahrheitstabelle}
\begin{tabular}{|l|l||l|l|l|l|l|l|l|}
  \hline
  & & Nicht &  & Und & Oder & Implikation & Äquivalenz & NAND \\
  \hline
  $A$ & $B$ & $\neg A$ & $\neg (\neg A)$ & $A \wedge B$ & $A \vee B$ & $A \Rightarrow B$ & $A \Leftrightarrow B$ & $A \uparrow B$ \\
  \hline
  \hline
  w & w & f & w & w & w & w  & w  & f \\
  w & f & f & w & f & w & f  & f  & w \\
  f & w & w & f & f & w & w  & f  & w \\
  f & f & w & f & f & f & w  & w  & w \\
  \hline
\end{tabular}
\begin{itemize}
  \item In der Wahrheitstabelle gilt: w = wahr und f = falsch.
  \item Eine Aussage, die in jeder Zeile der Wahrheitstafel falsch ist, heisst Kontradiktion.
  % \item Aussagen wie $A \vee B$ sind Aussagenlogische Formeln.
\end{itemize}

\subsubsection{Beispiel}
\( \neg (\neg \colorbox{green}{$(A \Rightarrow B)$} \Rightarrow \neg \colorbox{yellow}{$(B \Rightarrow A)$}) \Rightarrow \colorbox{pink}{$A \wedge B$} \)

\begin{tabular}{|l|l||l|l|l|l|l|l|l|l|l}
  \hline
  1 & 2 & 3 & 4 & 5 & 6 & 7 & 8 & 9 & 10 \\
  \hline
  $A$ & $B$ & \cellcolor{green} $A \Rightarrow B$ & $\neg 3$ & \cellcolor{yellow} $B \Rightarrow A$ & $\neg 5$ & $4 \Rightarrow 6$ & $\neg 7$ & \cellcolor{pink} $A \wedge B$ & $8 \Rightarrow 9$\\
  \hline
  \hline
  w & w & w & f & w & f & w & f & w & w \\
  w & f & f & w & w & f & f & w & f & f \\
  f & w & w & f & f & w & w & f & f & w \\
  f & f & w & f & w & f & w & f & f & w \\
  \hline
\end{tabular}

\subsection{Hinreichend und notwendig}
Die Implikation $A \Rightarrow B$ bedeutet:
\begin{itemize}
  \item Wenn A wahr ist, dann ist auch B wahr. A ist eine \emph{hinreichende} Bedingung für B.
  \item A kann nicht wahr sein, wenn B falsch ist. B ist eine \emph{notwendige} Bedingung für A.
\end{itemize}

\subsection{Aussagenlogische Formeln}
\begin{description}
  \item[Kommutativität] $A \wedge B \Leftrightarrow B\wedge A$ und $A \vee B \Leftrightarrow B\vee A$
  \item[Assoziativität] $A \wedge (B \wedge C) \Leftrightarrow (A \wedge B) \wedge C$
    und $A \vee (B \vee C) \Leftrightarrow (A \vee B) \vee C$
  \item[Distributivität] $A \wedge (B \vee C) \Leftrightarrow (A \wedge B) \vee (A \wedge C)$\\
    und $A \vee (B \wedge C) \Leftrightarrow (A \vee B) \wedge (A \vee C)$
  \item[Satz de Morgan] $\neg(A \wedge B) \Leftrightarrow \neg A \vee \neg B$
    und $\neg(A \vee B) \Leftrightarrow \neg A \wedge \neg B$
  \item[Aufeinanderfolgende Implikationen] Wenn $A \Rightarrow B$ und $B \Rightarrow C$ \\ kann man auch schreiben
    $(A \Rightarrow B \Rightarrow C) \Leftrightarrow (A \Rightarrow B) \wedge (B \Rightarrow C)$.
\end{description}
Bei der Elimination von Klammern sind dies die Prioritäten:
\begin{enumerate}
  \item Klammern ($(\dots),[\dots],\{\dots\}$)
  \item Negation ($\neg$)
  \item Konjunktion ($\wedge$) und Disjunktion ($\vee$)
  \item Implikation ($\Rightarrow$) und Äquivalenz ($\Leftrightarrow$)
\end{enumerate}

\subsection{Normalform}
Eine Vollkonjunktion ist ein boolescher Ausdruck, in dem alle Variablen genau einmal vorkommen und durch $\wedge$ (konjunktiv) oder $\vee$ (disjunktiv) verbunden sind. Dabei dürfen die Variablen auch negiert auftreten.
Das wird an folgendem Beispiel gezeigt:

\begin{tabular}{|l|l|l||l|}
  \hline
  $X$ & $Y$ & $Z$ & $C$ \\
  \hline
  w & w & w & f \\
  w & w & f & f \\
  w & f & w & w \\
  w & f & f & w \\
  f & w & w & w \\
  f & w & f & w \\
  f & f & w & f \\
  f & f & f & w \\
  \hline
\end{tabular}


\subsubsection{Negationsnormalform}
  Bei der Negationsnormalform dürfen Negationen ($\neg$) nur direkt vor
  einer Variable (und nicht vor einer Klammer) stehen:
  \[\neg (A \vee B) \Leftrightarrow \neg A \wedge \neg B\]
\subsubsection{Verallgemeinerte Konjunktion}
  Alle Variabeln werden mit der Konjunktion ($\wedge$) verbunden.
  Die einzelnen Variabeln liegen dabei immer in der Negationsnormalform vor:
  \[ X \wedge \neg Y \wedge Z\]

\subsubsection{Disjunktive Normalform}
  Die disjunktive Normalform ist eine Verbindung der verallgemeinerten
  Konjunktionen mit einer Disjunktion ($\vee$).
  \highlight{Dabei werden die wahren Werte der Wahrheitstabelle ausgewertet.}
  \[ (X \wedge \neg Y \wedge Z) \vee (X \wedge \neg Y \wedge \neg Z) \vee (\neg X \wedge Y \wedge Z)
    \vee (\neg X \wedge Y \wedge \neg Z) \vee (\neg X \wedge \neg Y \wedge \neg Z)\]
\subsubsection{Verallgemeinerte Disjunktion}
  Alle Variabeln werden mit der Disjunktion ($\vee$) verbunden.
  Die einzelnen Variabeln liegen dabei immer in der Negationsnormalform vor:
  \[ X \vee \neg Y \vee Z\]
\subsubsection{Konjunktive Normalform}
  Die konjunktive Normalform ist eine Verbindung von verallgemeinerten Disjunktion
  mit einer Konjunktion ($\wedge$). \highlight{Dabei werden die falschen Werte
  der Wahrheitstabelle ausgewertet.}
  \[ A \Leftrightarrow \neg(X \wedge Y \wedge Z) \wedge \neg(X \wedge Y \wedge \neg Z) \wedge \neg(\neg X \wedge \neg Y \wedge Z) \]
  Daraus macht man noch die Negationsnormalform (mit dem Satz de Morgan):
  \[ A \Leftrightarrow (\neg X \vee \neg Y \vee \neg Z) \wedge (\neg X \vee \neg Y \vee Z) \wedge (X \vee Y \vee \neg Z) \]
  Jetzt hat man in den Klammern die verallgemeinerte Disjunktion und A liegt in der konjunktiven Normalform vor.


\subsection{Aussageformen und Prädikate}
  \begin{itemize}
    \item Der Wahrheitswert einer Aussageform hängt von einer oder mehreren Variablen ab.
    \item Aussageformen sind unbestimmt, weil nicht klar ist, welcher Wert
      für eine Variable eingesetzt wird.
    \item Der Wert, welcher für eine Aussageform eingesetzt wird, heisst Subjekt.
    \item Ein Subjekt kann zulässig (Aussage ist wahr oder falsch) oder unzulässig
      (Aussage ist nicht auswertbar) sein.
    \item Aussagen und Aussageformen bestehen aus dem Subjekt (Variable)
      und dem Prädikat (Beschreibung, Eigenschaft).
  \end{itemize}

\subsection{Quantoren}
  \begin{itemize}
    \item Der Allquantor ($\forall$) sagt, dass eine Aussage für alle
      Elemente gelten soll.
    \item Der Existenzquantor ($\exists$) sagt, dass eine Aussage für
      mindestens ein Element gelten soll.
    \item Gibt es einen Quantor, sind die Variabeln nicht mehr frei wählbar,
      sondern an den Quantor gebunden.
  \end{itemize}
  Das wird an folgendem Beispiel erläutert:
  \[R(x) : \text{Der Weg } x \text{ aus der Menge } W \text{ aller Wege führt nach Rom.} \]
  \begin{itemize}
    \item Alle Wege führen nach Rom: $\forall x \in W: R(x)$
    \item Nicht alle Wege führen nach Rom: $\neg \forall x \in W: R(x)
    \Leftrightarrow \exists x \in W: \neg R(x)$
    \item Kein Weg führt nach Rom: $\neg \exists x \in W: R(x)$
    \item Alle Wege führen nicht nach Rom: $\forall x \in W: \neg R(x)$
  \end{itemize}
  Wenn nicht alle Wege nach Rom führen, gibt es mindestens einen Weg,
    der Nicht nach Rom führt:
  \[ \neg (\forall x \in W: R(x)) \Leftrightarrow \exists x \in W: \neg R(x) \]
  Man kann aber nichts darüber aussagen, ob überhaupt ein Weg nach Rom führt.

\subsection{Natürliche Zahlen}
  Null ist eine natürliche Zahl und jede natürliche Zahl hat einen
    Nachfolger (ausser der Null).
  Wir können die Nachfolger von $n$ mit $s(n)$ darstellen:
  \[ 0, s(0), s(s(0)), s(s(s(0))), \dots \]
  Das kann später noch nützlich sein.

%%%%%%%%%%%%%%%%%%%%%%%%%%%%%%%%%%%%%%%%%%%%%%%%%%%%%%%%%%%%%%%%%%%%%%%%
\section{Beweisen}
\subsection{Allgemeine Beweistechniken}
Wir haben folgende Voraussetzung:
  \[ a \in \mathbb{R}, b \in \mathbb{R}, a > 0, b > 0 \]
  Folgende Behauptung soll geprüft werden:
  \[ a^2 < b^2 \Rightarrow a < b \]
  Dabei ist
  \[ A: a^2 < b^2 \text{ und } B: a < b \text{ und } C: (A \Rightarrow B) \]
  \subsubsection{Direkter Beweis}
  Behauptung wird anhand allgemein geltenden Grundlagen abgeleitet und man versucht die Aussage $A(n) \Rightarrow B(n)$ direkt zu zeigen.
  Ist \[a^2 < b^2\], dann ist \[0 < b^2 - a^2\] was \[0 < (b + a)(b - a)
  \Rightarrow b > a\] bedeutet.
  \subsubsection{Indirekter Beweis}
  Um zu zeigen 'wenn A, dann B' gilt, können wir auch sagen 'wenn nicht b, dann auch nicht a'.
  \[ (A \Rightarrow B) \Leftrightarrow (\neg B \Rightarrow \neg A) \]

\subsection{Vollständige Induktion}
Folgende Formel ist zu beweisen:
\[ S(n): \sum\limits_{k=1}^n k = \frac{1}{2} \cdot n \cdot (n + 1)\]
\subsubsection{Induktionsanfang / Induktionsverankerung}
Man prüft die Formel für die erste Zahl der Reihe. In unserem Fall ist
das $n = 1$:
\[ n = 1: \sum\limits_{k=1}^1 k = 1 = \frac{1}{2} \cdot 1 \cdot (1 + 1) = 1\]
Die Summenformel git also für n = 1.

\subsubsection{Induktionsschritt}
\textbf{a) Induktionsannahme} \\
Das ist die ursprüngliche Formel:
\\ $S(n): \sum\limits_{k=1}^n $ \colorbox{red}{$ k $} $ = $ \colorbox{green}{$\frac{1}{2} \cdot n \cdot (n + 1)$}

\textbf{b) Induktionsbehauptung} \\
Man geht davon aus, dass die Formel auch für die darauf folgende Zahl
gilt. Deshalb erhöhnt man $n$ überall um 1:
 \[ S(n + 1): \sum\limits_{k=1}^{n+1} k = \frac{1}{2} \cdot (n + 1) \cdot (n + 2)\]
(Quasi \texttt{s/n/n+1/g}) \\
Die Induktionsbehauptung ist jetzt mit dem Induktionsbeweis zu beweisen.

\textbf{c) Induktionsbeweis} \\
Dann muss man die linke Seite der Induktionsbehauptung nehmen und in zwei Summen Teilen,
damit man die Induktionsannahme einsetzen kann:
 \[ \sum\limits_{k=1}^{n+1} k = \left( \sum\limits_{k=1}^{n} k \right) + n + 1 \]
Das $n + 1$ auf der Rechten Seite kommt vom linken Summenzeichen.
Das $n + 1$, bedeutet noch ein $k$ (von der nächsten Zahl) dazu addieren.

Dann setzt man die Induktionsannahme beim Summenzeichen nach dem
Gleichheitszeichen ein:
\\ $\sum\limits_{k=1}^{n+1} k = $ \colorbox{green}{$\frac{1}{2} \cdot n \cdot (n + 1)$}$ + $ \colorbox{red}{$ (n + 1) $}

Multiplizieren und $\frac{1}{2}$ ausklammern:
 \[ \sum\limits_{k=1}^{n+1} k = \frac{1}{2} \cdot (n^2 + n + 2n + 2) =
 \frac{1}{2} \cdot (n^2 + 3n + 2) \]

Klammern als Produkt schreiben:
 \[ \sum\limits_{k=1}^{n+1} k = \frac{1}{2} \cdot (n + 1) \cdot (n + 2)\]
Somit ist man wieder bei der gleichen Formeln der Induktionsbehauptung.

\subsection{Rekursionen}
Eine Reihe ist eine Summe von Folgegliedern ($1 + 2 + 3 + ...$). Eine Folge ist eine Menge
von Zahlen, in spezieller Reihenfolge:
\[ (a_k)_{k...n} := (a_0, a_1, a_2, ..., a_{n-1}, a_n) \]
\[ \Rightarrow (a_k)_{n \in \mathbb{N}} := (a_0, a_1, a_2, a_3, a_4, ...) \]
\subsubsection{Direkte Angabe}
Man kann für jedes Folgeglied angeben, wie es berechnet wird:
\[ a_n = 2^n \]
\[ a_1 = 2; a_2 = 4; a_3 = 8, ... \]
\subsubsection{Rekursive Angabe}
Man gibt das erste Folgeglied an, sowie eine Rekursionsformel,
wie man das nächste Glied berechnet.
\[a_0 = 1 \text{ und } a_n = a_{n-1} \cdot 2 \]
\[ \Rightarrow a_1 = 2; a_2 = 4; a_3 = 8, ... \]

%%%%%%%%%%%%%%%%%%%%%%%%%%%%%%%%%%%%%%%%%%%%%%%%%%%%%%%%%%%%%%%%%%%%%%%%
\section{Mengen, Relationen, Abbildungen}

\subsection{Mengen, Teilmengen, Potenzmengen}
\begin{itemize}
  \item Mengen: $\{5, 23, 42\}$ oder $\{1, 4, 9, 16, 25, ...\}$ oder
  $\{x \in \mathbb{N} | x^2 = 1 \}$
  \item Teilmenge/Inklusion: A ist Teilmenge von B: $A \subset B$ (Ganz A ist enthalten in B)
  \highlight{\item Leere Menge: $\emptyset$ oder $\{ \}$ (Die leere Menge ist
    Teilmenge jeder Menge!)}
  \item Mächtigkeit oder Kardinalität $|M|$: Anzahl der Elemente: Ist $M = \{a, b, c\}$, dann ist $|M| = 3$
  \item Potenzmenge $P(M)$: Menge aller Teilmengen von einer Menge $M$:
  \\ Die Kardinalität der Potenzmenge ist zwei hoch der Kardinalität der Ausgangsmenge:
  \\ Ist die Menge $M = \{a, b\}$, dann ist die Potenzmenge $P(M) = \{\{\},
  \{a\}, \{b\}, \{a, b\}\}$ $(|P(M)| = 2^2 = 4)$

\end{itemize}

\subsection{Vereinigung und Durchschnitt}
\subsubsection{Schreibweise}
\begin{itemize}
  \item Vereinigung: $A \cup B = \{ x | x \in A \text{ oder } x \in B \}$
  \item Durchschnitt / Schnittmenge: $A \cap B = \{ x | x \in A \text{ und } x \in B \}$
  \item Differenz: $A \backslash B = \{ x | x \in A \text{ und } x
    \notin B \} = A \cap \overline B$
  \item Komplement: $\overline A = \{ x | x \notin A \}$
\end{itemize}

\subsubsection{Gesetze}
\begin{itemize}
  \item Idempotenzgesetz: $A \cup A = A$ und $A \cap A = A$
  \item Kommutativität: $A \cup B = B \cup A$ und $A \cap B = B \cap A$
  \item Assoziativität: $A \cup (B \cup C) = (A \cup B) \cup C = A \cup
  B \cup C$ \\ und $A \cap (B \cap C) = (A \cap B) \cap C = A \cap B \cap C$
  \item Distributivität: $A \cup (B \cap C) = (A \cup B) \cap (A \cup C)$
    und $A \cap (B \cup C) = (A \cap B) \cup (A \cap C)$
\end{itemize}
Zusammenhang der Inklusion/Teilmenge mit der Vereinigung und Durchschnitt:
\[ A \subset B \Leftrightarrow (A \cap B) = A \Leftrightarrow (A \cup B)
= B \]

\subsection{Komplement und Differenz}
Das Komplement kann man mit der Menge $A$ und $B$ und der Obermenge $M$
beschreiben:
\[\overline A = \{ x \in M | x \notin A\} \]
\[ \overline{A \cup B} = \overline A \cap \overline B \]
\[ A \cap \overline{A \cup B \cup C} = A \cap (\overline A \cap
  \overline B \cap \overline C) = (A \cap \overline A) \cup (B \cap
  \overline C) = \{\}\]

Bei der Differenz gilt folgendes:
\[ A \backslash B = \{ a \in A | a \notin B \}\]
\[ A \backslash B = A \cap \overline B \]
\[ A \backslash (A \backslash B) = A \backslash (A \cap \overline B) = A
  \cap (\overline{A \cap \overline B}) = A \cap (\overline A \cup B) = (A
  \cap \overline A) \cup (A \cap B) = \{\}\]

\subsection{Venn-Diagramm}
In einem Venn-Diagramm können die Mengen grafisch dargestellt werden.

\subsection{Kartesische Produkte}
Im Gegensatz zu den Mengen spielt bei den Kartesischen Produkten die
Reihenfolge der Elemente eine Rolle.

Ein geordnetes Paar nennt sich auch 2-Tupel und ist wie folgt definiert:
\[ (a, b) = (a_0, b_0) \Leftrightarrow a = a_0 \text{ und } b = b_0 \]

\highlight{Das Kartesische Produkt von zwei Mengen $A$ und $B$ wird definiert als
Menge aller geordneten Paare, deren erste Komponente aus der Menge $A$
stammt und die zweite Komponente aus der Menge $B$}:
\[ A \times B = \{(a,b)|a \in A, b \in B \}\]


Dasselbe gilt auch für Kartesische Produkte mit drei Faktoren:
\[ A \times B \times C = \{(a,b,c)|a \in A, b \in B, c \in C \}\]

Dasselbe gilt auch für Kartesische Produkte mit beliebig vielen Faktoren:
\[ A_1 \times A_2 \times ... \times A_n = \{(a_1,a_2,...,a_n)|a_i \in
A_i \text{ für } i = 1, ..., n \}\]
Sind alle Faktoren gleich, kann man die Potenzschreibweise verwenden:
\[ A^n = \underbrace{A \times A \times ... \times A}_{\text{n Faktoren}} \]
\subsection{Relationen}
Eine n-stellige Relation $R$ zwischen den nichtleeren Mengen $A_1, A_2,
..., A_n$ ist eine Teilmenge des kartesischen Produktes $A$.
\[ \prod \limits_{i=1}^{n}A_i = A_i \times A_i \times ...  \times A_i \]
\[ R \subset A = \prod \limits_{i=1}^{n}A_i \]

\subsection{Abbildungen}
\begin{itemize}
  \item Injektiv: \emph{Jedem} Element der Definitionsmenge wird ein
    \emph{anderes} Element der Zielmenge zugeordnet.
  \item Surjektiv: \emph{Jedem} Element der Definitionsmenge wird ein
    Element der Zielmenge zugeordnet und alle Elemente der Zielmenge werden
    mindestens einmal erreicht.
  \item Bijektiv (Injektiv und Surjektiv): \emph{Jedem} Element der
  Definitionsmenge wird ein \emph{anderes} Element der Zielmenge zugeordnet und
  alle Elemente der Zielmenge werden erreicht.
\end{itemize}
\subsection{Mächtigkeit von Mengen}
Zwei Mengen $A$ und $B$ heissen gleich mächtig, wenn es eine bijektive
Abbildung von $A$ nach $B$ gibt.

%%%%%%%%%%%%%%%%%%%%%%%%%%%%%%%%%%%%%%%%%%%%%%%%%%%%%%%%%%%%%%%%%%%%%%%%
\section{Vektoren und Vektorräume}

\subsection{Begriffe}
\subsubsection{Vektoren}
Einen n-Tupel $(x_1, x_2, x_3, ..., x_n)$ aus dem $\mathbb{R}^n$ kann
als Vektor dargestellt werden:
\[ \vec x_n = \left(
    \begin {array} {c}
        x_1 \\
        x_2 \\
        x_3 \\
        \vdots \\
        x_n
    \end {array}
  \right) \]

\subsubsection{Vektoren aus zwei Punkten berechnen}
Der Vektor von Punkt $Q = (3, 0)$ bis $P = (-1, 3)$ berechnet sich so:
\[ \overline{QP} = \begin {pmatrix} P_1 - Q_1 \\ P_2 - Q_2 \\ \end {pmatrix}
  = \begin {pmatrix} -1 - 3 \\ 0 - (-3) \\ \end {pmatrix}
  = \begin {pmatrix} -4 \\ 3 \\ \end {pmatrix} \]

\subsubsection{Matrix, Matritzen}
Eine $n \times m$ Matrix oder eine Matrix vom Typ $(n, m)$ hat $n$ Zeilen
und $m$ Spalten, wird mit einem Grossbuchstaben beschrieben und sieht so aus:
\[ A = \begin {pmatrix}
  a_{11} & a_{12} & a_{13} & \hdots & a_{1m} \\
  a_{21} & a_{22} & a_{23} & \hdots & a_{2m} \\
  \vdots  & \vdots  & \vdots  &     & \vdots  \\
  a_{n1} & a_{n2} & a_{n3} & \hdots & a_{nm} \\
\end {pmatrix} \]
Die Zahlen heissen Koeffizienten und werden mit zwei Indizes
geschrieben. Der erste ist der Zeilenindex ($n$), der zweite ist der
Spaltenindex ($m$).

\textbf{Quadratische Matrix:} Eine Matrix ist quadratisch, wenn $n = m$ ist.
Die quadratische Matrix hat also gleich viele Spalten wie Zeilen:
\[ M = \begin {pmatrix}
 1  & 2 & 3  \\
 4  & 5 & 6  \\
 7  & 8 & 9  \\
\end {pmatrix} \]

\textbf{Diagonalmatrix:} Eine Diagonalmatrix ist quadratisch und zudem
alle Koeffizienten $0$ ausser die Diagonale:
\[ M = \begin {pmatrix}
 5  & 0  & 0  & 0  & 0  \\
 0  & 23  & 0  & 0  & 0  \\
 0  & 0  & 50  & 0  & 0  \\
 0  & 0  & 0  & 42  & 0  \\
 0  & 0  & 0  & 0  & 6  \\
\end {pmatrix} \]

\textbf{Einheitsmatrix:} Eine Einheitsmatrix ist eine Diagonalmatrix,
welche aus lauter 1 besteht:
\[ M = \begin {pmatrix}
 1  & 0  & 0  & 0 \\
 0  & 1  & 0  & 0 \\
 0  & 0  & 1  & 0 \\
 0  & 0  & 0  & 1 \\
\end {pmatrix} \]

\textbf{Dreiecksmatrix:} Eine (obere) Dreiecksmatrix ist eine Diagonalmatrix,
in welcher oberhalb der Diagonale nicht aus Nullen besteht:
\[ M = \begin {pmatrix}
 1  & 2  & 3  & 4 \\
 0  & 9  & 8  & 5 \\
 0  & 0  & 7  & 6 \\
 0  & 0  & 0  & 1 \\
\end {pmatrix} \]

\subsubsection{Rang}
Anzahl Zeilen der Matrix A, die bei der Lösung eines linearen
Gleichungssystems mit dem Gauss-Algorithmus nicht zu Nullzeilen werden.
Die Matrix
\[ M = \begin {pmatrix}
 5  & 0  & 0 \\
 0  & 23  & 0 \\
 0  & 0  & 0 \\
\end {pmatrix} \]
hat also den Rang 2.

\subsubsection{Dimension}
Die Anzahl der Elemente einer Basis heisst die Dimension des Vektorraums, kurz \( \text{dim}(V) \). Die Elemente der Basis werden gebraucht, um den Lösungsraum des homogenen Gleichungssystems zu bestimmen (die \highlight{linear abhängigen} Vekoren durch die \highlight{linear unabhängignen} Vektoren berechnen).
\[ \text{dim}(\text{Lös}(A, \vec{0})) = n - Rg(A)\]
\( n \) steht für Anzahl Unbekannte.

\subsubsection{Linearkombination}
\begin{itemize}
  \item Alle Linearkombinationen einer Menge von Vektoren bilden einen
  Vektorraum
  \item Aus einem linear unabhängigen Erzeugendensystem lässt sich kein Vektor
    weglassen, ohne den erzeugten Vektorraum zu verkleinern
  \item Einige Vektoren (linear abhängige) lassen sich als Linearkombinationen von anderen
  Vektoren darstellen
\end{itemize}

\subsubsection{Lineare Abhängigkeit}
Vektoren sind linear unabhängig, wenn keiner der Vektoren sich
als Linearkombination der jeweils anderen Vektoren darstellen lässt
(sie dürfen also z. B. nicht in die selbe Richtung zeigen.)

Das heisst, dass Vektoren linear unabhängig sind, wenn aus $\lambda_1 \cdot \vec{v_1} +
\lambda_2 \cdot \vec{v_2} + ... + \lambda_n \cdot \vec{v_n} = \vec{0}$
folgt, dass $\lambda_1 = \lambda_2 = ... = \lambda_n = 0$

Wenn es nach der Gauss-Elimination Nullzeilen gibt, sind die Vektoren
linear abhängig (= nicht linear unabhängig). \highlight{Es gibt nur so viele linear
unabhängige Vektoren, wie es keine Nullzeilen gibt (=Rang):}
\[ M = \begin {pmatrix}
  1  &  0  &  -1  &  0  &  1  \\
  0  &  1  &   2  &  3  &  4  \\
  0  &  0  &   0  &  0  &  0  \\
  0  &  0  &   0  &  0  &  0  \\
\end {pmatrix} \]
\begin{itemize}
  \item Die fünf Vektoren (= Spalten der Matrix) sind linear abhängig
  \item \( \text{Rg}(M) = 2\) (Nur zwei der fünf Vektoren sind linear unabhängig)
  \item \( \text{dim}(M) = 3\)
\end{itemize}
\[ x_1 = x_3 - x_5 \]
\[ x_2 = -2x_3 - 3x_4 - 4x_5 \]
\\ Die Komponenten $x_3$, $x_4$ und $x_5$ werden gewählt und erhält:
\[
  \vec{v_1} = \begin {pmatrix} 1 \\ -2 \\ \color{red}{1} \\ 0 \\ 0 \end {pmatrix},
  \vec{v_2} = \begin {pmatrix} 0 \\ -3 \\ 0 \\ \color{red}{1} \\ 0 \end {pmatrix},
  \vec{v_3} = \begin {pmatrix} -1 \\ -4 \\ 0 \\ 0 \\ \color{red}{1} \end {pmatrix}
\]
% TODO evtl. ergänzen aus Skirpt Algebra1-32 Seite 21

\subsubsection{Erzeugendensystem}
Ein Erzeugendensystem $E$ spannt eine Ebene / einen Raum der
Dimension $n$ auf:
\[ E = \{ \vec{v_1}, \vec{v_2}, ..., \vec{v_n}\} \]
Die Vektoren $\vec{v_1}$ bis $\vec{v_n}$ müssen nicht zwingend linear
unabhängig sein.

\subsubsection{Basis}
Eine Basis $B$ ist eine Menge von linear unabhängigen Vektoren $B =
\{\vec{b_1}, \vec{b_2}, ..., \vec{b_n}\}$, welche einen Vektorraum $V$
aufspannen. Jeder Vektor $\vec{x} \in V$ besitzt eindeutige Koordinaten
$x_1, x_2, ..., x_n$ bezüglich der Basis $B$. Das heisst $\vec{x} = x_1
\cdot \vec{b_1} + x_2 \cdot \vec{b_2} + ... + x_n \cdot \vec{b_n}$ ist
eindeutig.
Die Vektoren $\vec{v_1}$ bis $\vec{v_n}$ sind zwingend linear
unabhängig. Keiner der Vektoren ist also ein Vielfaches eines anderen
Vektoren.

\subsubsection{Einheitsbasis}
Da es unendlich viele Basen eines Vektorraums gibt, hat man die
Einheitsbasis $B_E = \{ \vec{e_1}, \vec{e_2}, ..., \vec{e_n}\}$
definiert.  Die Einheitsbasis / Normalbasis / kanonische Basis besteht
aus Einheitsvektoren, die linear unabhängig sind:
\[ B_E = \vec{e_1} = \begin{pmatrix} 1 \\ 0 \\ 0 \\ \vdots \\ 0 \end{pmatrix},
  \vec{e_2} = \begin{pmatrix} 0 \\ 1 \\ 0 \\ \vdots \\ 0 \end{pmatrix},
  \vec{e_3} = \begin{pmatrix} 0 \\ 0 \\ 1 \\ \vdots \\ 0 \end{pmatrix},
  ...,
  \vec{e_n} = \begin{pmatrix} 0 \\ 0 \\ 0 \\ \vdots \\ 1 \end{pmatrix} \]

\begin{itemize}
  \item Daraus erkennt man, dass der Betrag $|\vec{e_i}|$ immer $1$ ist.
  \item Beim Skalarprodukt von zwei Einheitsvektoren der Normalbasis
    gilt
    \[ \vec{e_i} \cdot \vec{e_j} =
       \begin{cases}
         0 & \text{ wenn } \vec{e_i} \ne \vec{e_j} \\
         1 & \text{ wenn } \vec{e_i} = \vec{e_j} \\
       \end{cases}
    \]
\end{itemize}


\subsection{Rechnen mit Vektoren}
Vektoren können addiert werden:
\[ \vec x_n + \vec y_n =
  \left( \begin {array} {c} x_1 \\ x_2 \\ \vdots \\ x_n \end {array}
  \right) +
    \left( \begin {array} {c} y_1 \\ y_2 \\ \vdots \\ y_n \end {array}
    \right) =
    \left( \begin {array} {c} x_1 + y_1 \\ x_2 + y_2 \\ \vdots \\ x_n + y_n \end {array} \right) \]
Vektoren können multipliziert werden:
\[
23 \cdot \vec x = 23 \cdot
  \left( \begin {array} {c} x_1 \\ x_2 \\ \vdots \\ x_n \end {array}
  \right) =
    \left( \begin {array} {c} 23 \cdot x_1 \\ 23 \cdot x_2 \\ \vdots
    \\ 23 \cdot x_n \end {array} \right)
\]
Der Definitionsbereich wird mit $\vec x_n \in \mathbb{R}^n$ angegeben.

\subsubsection{Weitere Rechenregeln für Vektoren}
\begin{itemize}
  \item $\vec 0$ ist das neutrale Element: $\vec v + \vec 0 = \vec v$
  \item $- \vec v$ ist das inversive Element: $\vec v + (- \vec v) = 0$
  \item Es gilt die Assoziativität: $(\vec u + \vec v ) + \vec w = \vec u + (\vec v + \vec w)$
  \item Es gilt die Kommutativität: $\vec u + \vec v = \vec v + \vec u$
  \item Man kann Skalarkörper ($s$) ausmultiplizieren oder ausklammern:
    $s (\vec u + \vec v) = s \cdot \vec u + s \cdot \vec v$
\end{itemize}

\subsubsection{Matrix mal Vektor}
Eine Matrix kann mit einem Vektor multipliziert werden (Zeile der Matrix $\cdot$
Spalte vom Vektor):
\[ A \cdot \vec x = \vec b = \begin {pmatrix}
  a_{11} & a_{12} & a_{13} \\
  a_{21} & a_{22} & a_{23} \\
  a_{31} & a_{32} & a_{33} \\
\end {pmatrix} \cdot
  \left( \begin {array} {c} x_1 \\ x_2 \\ x_3 \end {array} \right) =
  \left( \begin {array} {c}
            a_{11}x_1 + a_{12}x_2 + a_{13}x_3 \\
            a_{21}x_1 + a_{22}x_2 + a_{23}x_3 \\
            a_{31}x_1 + a_{32}x_2 + a_{33}x_3 \\
         \end {array}
  \right) =
  \left( \begin {array} {c} b_1 \\ b_2 \\ b_3 \end {array} \right)
     \]
Zudem gilt folgendes Gesetz:
\[ A \cdot (\vec v + \vec w) = A \cdot \vec v + A \cdot \vec w \]

\subsection{Lineare Gleichungssysteme}
\subsubsection{Begriffe}
\begin{itemize}
  \item Ein lineares Gleichungssystem ist homogen, wenn die rechte Seitefür alle Gleichungen $= 0$ ist.
  \item Ein lineares Gleichungssystem ist inhomogen, wenn die rechte
    Seite nicht immer $= 0$ ist. Spezielle Lösung = Rückwärts einsetzen;
    allgemeine Lösung = Rechte Seite gleich Null.
\end{itemize}

\subsubsection{Erweiterte Koeffizientenmatrix}
Das (inhomogene) Gleichungssystem
\begin{align}
   2x_1 - x_2 - 2x_3 = 0 \\
   -1x_1 + 2x_2 + 3x_3 = 3 \\
   3x_1 - x_3 = 3
\end{align}
kann auch als erweiterte Koeffizientenmatrix geschrieben werden:
\[ \begin{array}{rrr|r}
2   &   -1    & -2    & 0   \\
-1  &   2     & 3     & 3   \\
3   &   0     & -1    & 3   \\
\end {array} \]

\subsubsection{Gauss-Algorithmus / Gauss-Elimination}
Mit der Gauss-Elimination macht man die Zahlen links unter der Diagonale
der erweiterten Koeffizientenmatrix zu Nullen. Dabei darf man folgende
Aktionen durchführen:
\begin{itemize}
  \item Zeilen vertauschen: gibt keine Probleme
  \item Spalten vertauschen: ok, \highlight{aber beim Lösungsvektor wieder
    zurücktauschen!}
  \item Multiplikation einer Gleichung mit einem Skalar
  \item Addition einer Gleichung zu einer anderen Gleichung
\end{itemize}
Nach den Umformungen kann das so aussehen:
\[ \begin{array}{rrr|r}
  -1 & 2  & 3  & 0 \\
  0 & 3  & 4  & 6 \\
  0 & 0  & 0  & 0 \\
\end {array} \]
Durch rückwärts Einsetzen kann man jetzt \emph{eine} Lösung bestimmen:
$x_3$ sieht man in der untersten Zeile, also ist $x_3 = 0$. Jetzt setzt
man $x_3$ in der zweiten Zeile ein, das ergibt $x_2 = 2$. Dasselbe macht
man für $x_1$ und man erhält folgenden Vektor:
\[ \vec x = \left( \begin{array} {c} 1 \\ 2 \\  0 \end{array} \right) \]
Setzt man diese Lösung ins erste Gleichungssystem ein, sieht man, dass
es mit diesen Lösungen funktioniert. Die Lösung ist auch für alle
Vielfachen der Lösung gültig.

\subsubsection{Lösungsmengen}

Ein Gleichungssystem kann keine Lösung haben:
\begin{align}
  x_1 + x_2 = 1 \\
  x_1 + x_2 = 2
\end{align}

Oder unendlich viele:
\[ x_1 + x_2 = 2 \] Dabei ist die Lösungsmenge für $x_1 = t$ und $x_2 =
2 - t$.

Ist die letzte Zeile eine Nullzeile, gibt es auch unendlich viele
Lösungen:
\[ \begin{array}{rrrr}
1  &   -2     & -3     & -3   \\
0   &   3     & 4     & 6   \\
0   &   0     & 0     & 0   \\
\end {array} \]
Eine spezielle Lösung gibt es durch Rückwärtseinsetzen mit $x_3 = 0
\Rightarrow x_2 = 2 \Rightarrow = 1$. Das ist der Aufhänger:
\[ \vec v = \left( \begin {array} {c} 1 \\ 2 \\ 0 \\ \end {array} \right) \]

Die allgemeine Lösung ergibt sich dadurch, dass man die rechte Seite
gleich Null setzt:
\[ \begin{array}{rrrr}
1  &   -2     & -3     & 0 \\
0   &   3     & 4     & 0 \\
0   &   0     & 0     & 0 \\
\end {array} \]
Die dritte Zeile ist eine Nullzeile. Die zweite Zeile heisst $3x_2 +
4x_3 = 0$. Das hat unendlich viele Lösungen, z. B. $x_2 = 4$ und $x_3 =
-3$. Dann folgt aus der ersten Zeile wegen $x_1 - 2x_2 - 3x_3 = 0$, dass
$x_1 = -1$ ist. Eine Lösung des homogenen Systems ist also:
\[ \vec r = \left( \begin {array} {c} -1 \\ -4 \\ 3 \\ \end {array} \right) \]
Alle Lösungen des inhomogenen Gleichungssystems ergeben sich als Summe
einer speziellen Lösung des inhomogenen Gleichungssystems plus alle
Lösungen des homogenen Gleichungssystems:

\[ \text{Lösung}(A, \vec{b}) = \left \{
    \vec{x} \in \mathbb{R}^3 \middle |
      \vec{x} = \left( \begin {array} {c} 1 \\ 2 \\ 0 \\ \end {array} \right) +
      \lambda \cdot \left( \begin {array} {c} 1 \\ -4 \\ 3 \\ \end {array} \right),
      \lambda \in \mathbb{R}
  \right \} \]
\[ \text{Lösung} = \text{Spezielle Lösung} + \lambda \cdot \text{
allgemeine Lösung}, \lambda \in \mathbb{R} \]

\subsubsection{Lösen eines inhomogenen Gleichungssystems}
Inhomogene Gleichungssysteme haben die Rechte seite immer $\ne 0$ und
besitzen eine spezielle und eine allgemeine Lösung.

Das Gleichungssystem
\begin{align}
  x_1 + x_2 + 3 \cdot x_3 = 4 \\
  2 \cdot x_1 + x_2 - x_3 = -5 \\
  3 \cdot x_1 + 2 \cdot x_2 + 2 \cdot x_3 = -1
\end{align}
ergibt die Koeffizientenmatrix
\[ \begin {pmatrix}
  1       & 1       & 2  & 4 \\
  2       & 1       & -1  & -5 \\
  3       & 2       & 2  & 1 \\
\end {pmatrix} \]
Nach der Gauss-Elimination hat man folgende Matrix:
\[ \begin {pmatrix}
  1       & 1       & 3  & 4 \\
  0       & -1       & -7  & -13 \\
  0       & 0       & 0  & 0 \\
\end {pmatrix} \]
Daher ist $x_3 = 0$. Somit ist $x_2 = 13$ und $x_1 = 4 - 13 = -9$. Das
ist die spezielle Lösung.

Die allgemeine Lösung erhält man, wenn man $x_3 = 1$ und die
rechte Seite überall auf 0 setzt. Dann bekommt
man $x_2 = -7$ und $x_3 = 4$.

Die Lösungsmenge ist schlussendlich:
\[ \text{Lösung}(A, \vec{b}) = \left \{
    \vec{x} \in \mathbb{R}^3 \middle |
      \vec{x} = \left( \begin {array} {c} -9 \\ 13 \\ 0 \\ \end {array} \right) +
      \lambda \cdot \left( \begin {array} {c} 4 \\ -7 \\ 1 \\ \end {array} \right),
      \lambda \in \mathbb{R}
  \right \} \]
\[ \text{Lösung} = \text{Spezielle Lösung} + \lambda \cdot \text{
allgemeine Lösung}, \lambda \in \mathbb{R} \]

Wichtig: Hat man mehrere Nullzeilen, so muss für jede Nullzeile das
entsprechende $x_n$ auf 1 (und deas andere auf 0) gesetzt werden
und die Lösungsmenge erweitert
sich dann um einen Faktor mal die allgemeine Lösung von diesem $x_n$.

\subsection{Geraden und Ebenen}
\subsection{Definition}
Die Gerade
\[ G : \vec{p} = \vec{a} + s \cdot \vec{r} \]
ist definiert durch den Aufhänger $\vec{a}$, den Skalar/Parameter $s$
und den Richtungsvektor $\vec{r}$.

\subsubsection{Punkt auf Gerade}
Um zu prüfen, ob ein Punkt $B$ auf einer Geraden $g$ liegt, setzt man
die Gerade $g$ mit dem Punkt $B$ gleich und löst das Gleichungssystem.
% Um festzustellen, ob der Punkt $B$
% \[ B = \left( \begin{array}{c} -2 \\-1 \\3 \end{array} \right) \]
% auf der Gerade $g_1$
% \[ g : \vec{p} = \left( \begin{array}{c} 1 \\2 \\2 \end{array} \right)
%   + s \cdot \left( \begin{array}{c} 5 \\ -1 \\ 1 \end{array} \right) \]
% liegt, setzt man Punkt $B$ und die Gerade $g$ gleich:
% \[ \left( \begin{array}{c} 1 \\2 \\2 \end{array} \right)
%   + s \cdot \left( \begin{array}{c} 5 \\ -1 \\ 1 \end{array} \right)
% = \left( \begin{array}{c} -2 \\-1 \\3 \end{array} \right) \]

\subsubsection{Schnittpunkt zweier Geraden im $\mathbb{R}^3$}
Um den Schnittpunkt der Geraden $g_1$ und $g_2$ im $\mathbb{R}^3$ festzustellen, setzt man
diese gleich. Die Geraden heissen:
\[ g_1 : \vec{p} = \left( \begin{array}{c} 1 \\2 \\2 \end{array} \right)
  + s \cdot \left( \begin{array}{c} 5 \\ -1 \\ 1 \end{array} \right)
  \text{ und }
  g_2 : \vec{q} = \left( \begin{array}{c} -2 \\-1 \\3 \end{array} \right)
  + t \cdot \left( \begin{array}{c} -1 \\ 2 \\ -1 \end{array} \right) \]
Jetzt setzt man die zwei Geraden gleich:
\[ \left( \begin{array}{c} 1 \\2 \\2 \end{array} \right)
  + s \cdot \left( \begin{array}{c} 5 \\ -1 \\ 1 \end{array} \right)
  = \left( \begin{array}{c} -2 \\-1 \\3 \end{array} \right)
  + t \cdot \left( \begin{array}{c} -1 \\ 2 \\ -1 \end{array} \right) \]
Die Aufhänger kann man addieren und auf die rechte Seite nehmen. Die
Richtungsvektoren kommen auf die linke Seite.
\[
  s \cdot \left( \begin{array}{c} 5 \\ -1 \\ 1 \end{array} \right)
  \color{red}{-t \cdot} \color{black}{\left( \begin{array}{c} -1 \\ 2 \\ -1 \end{array} \right)}
  = \left( \begin{array}{c} -2 \\-1 \\3 \end{array} \right) -
  \left( \begin{array}{c} 1 \\2 \\2 \end{array} \right)
\]
Daraus folgt das Gleichungssystem \highlight{(Vorsicht mit den Vorzeichen!)},
welches man lösen kann:
\[ \begin {pmatrix}
  5 & \color{red}{1} & -3 \\
  -1 & \color{red}{-2} & -3 \\
  1 & \color{red}{1} & 1 \\
\end {pmatrix} \Leftrightarrow \begin {pmatrix}
  1 & 1 & 1 \\
  0 & -1 & -2 \\
  0 & -4 & -8 \\
\end {pmatrix} \]
Da in der ersten Spalte $s$ und in der zweiten Spalte $t$ ist (da das
Gleichungssystem so aufgestellt wurde), kommt man auf $t = 2$ und $s =
-1$. Jetzt kann man $s$ oder $t$ in $g_1$ bzw. $g_2$ einsetzen und man
erhält den Schnittpunkt $S = (-4, 3, 1)$.

Ist das Gleichungssystem nicht lösbar ($\text{Rang}(A,\vec{b}) >
\text{Rang}(A))$, gibt es keinen Schnittpunkt. Die Geraden sind dann
\emph{windschief}.

\subsection{Norm und Skalarprodukt}
\subsubsection{Betrag eines Vektors (Länge)}
Der Betrag eines Vektors, also die Länge, rechnet sich wie folgt:
\[ |\vec{v}| = \sqrt{\sum \limits_{i = 1}^n {v_i}^2} \]
Der Betrag vom Vektor $\vec{u}$ aus dem $\mathbb{R}^3$ ist also:
\[ |\vec{u}| = \left|\left( \begin{array}{c} 1 \\-2 \\2 \end{array}
\right)\right| = \sqrt{1^2 + 2^2 + 2^2} = \sqrt{1 + 4 + 4} = \sqrt{9} = 3 \]
Der Abstand zweier Vektoren ist der Betrag der Differenz der beiden
Vektoren. Der Abstand zwischen
\[ \vec{u} = \left( \begin{array}{c} 2 \\ -1 \end{array}\right)
\text{ und } \vec{v} = \left( \begin{array}{c} 3 \\ -4 \end{array}\right)
\text{ ist } |\vec{u} + \vec{v}| = \left|\left(\begin{array}{c} 1 \\ -3
\end{array}\right)\right| = \sqrt{1^2 + 3^2} = \sqrt{10} \]

Bei den Beträgen gelten folgende Rechenregeln:
\begin{itemize}
  \item Ist der Betrag eines Vektors $0$, ist der Vektor der Nullvektor:
    $|\vec{v}| = 0 \Leftrightarrow \vec{v} = \vec{0}$
  \item $|r \cdot \vec{v}| = |r| \cdot |\vec{v}|$
  \item Dreiecksungleichung: $|\vec{v_1} + \vec{v_2}| \le
    |\vec{v_1}| + |\vec{v_2}|$
\end{itemize}

\subsubsection{Normalenvektor}
Ein Normalenvektor ist ein Vektor, der \highlight{senkrecht} (orthogonal) auf einem Vektor, einer Geraden, einer Kurve, einer Fläche oder einer Ebene steht. Meistens wird der Normalenvektor zu einer Ebene gesucht. Für einen Normalenvektor ist einzig die Richtung entscheidend, daher gibt es jeweils beliebig viele Lösungen.
\\\emph{Das Skalarprodukt zwischen dem Vektor $\vec{v}$ und dem Normalenvektor $\vec{n_v}$ ist gleich 0.}
\\Der Normalenvektor zu einem Vektor:

\[ \text{aus } \vec{v} = \begin{pmatrix} 1 \\ 2 \\ 3 \\ 4 \end{pmatrix} \text{ergibt sich } \vec{n_v} = \begin{pmatrix} -2 \\ 1 \\ 0 \\ 0 \end{pmatrix} \]
\emph{Trick}: $v_1$ und $v_2$ auf vertauschen und eines davon mit $-1$
multiplizieren. Die restlichen $v_n$ auf Null setzen!
\\Der Normalenvektor zu einer Ebene (Ebene ist in Parameterform):
\[ E: \vec{v} = \begin{pmatrix} s_1 \\ s_2 \\ s_3 \end{pmatrix} + \lambda \cdot \begin{pmatrix} r_{11} \\ r_{12} \\ r_{13} \end{pmatrix} + \mu \cdot \begin{pmatrix} r_{21} \\ r_{22} \\ r_{23} \end{pmatrix} \]
\[ \text{Normalenvektor: } \vec{n_v} = \vec{r_1} \cdot \vec{r_2} \]

\subsubsection{Skalarprodukt}
Das Skalarprodukt ist das Produkt der Beträge zweier Vektoren mit dem
Kosinus des eingeschlossenen Winkels. Man erhält eine reelle Zahl.
\[ \vec{u} \cdot \vec{v} = |\vec{u}| \cdot |\vec{v}| \cdot \cos{(\phi)} \]
Das Skalarprodukt im $\mathbb{R}^n$ kann auch anders berechnet werden
(ohne Kosinus):
\[ \vec{u} \cdot \vec{v} = \sum \limits_{i = 1}^n u_i \cdot v_i \]
Das Skalarprodukt von den beiden Vektoren $\vec{u}$ und
$\vec{v}$ aus dem $\mathbb{R}^3$ mit
\[ \vec{u} = \left( \begin{array}{c} 2 \\ 6 \\ 3 \end{array}\right)
\text{ und } \vec{v} = \left( \begin{array}{c} 3 \\ 1 \\ -4 \end{array}\right) \]
ist also
\[ \vec{u} \cdot \vec{v} = 2 \cdot 3 + 6 \cdot 1 + 3 \cdot (-4) = 0 \]
Ist das Skalarprodukt $= 0$, stehen die Vektoren senkrecht zueinander!

Der Winkel der beiden Vektoren lässt sich mit folgender Formel
berechnen:
\[ \phi := \arccos{\left(\frac{\vec{u} \cdot \vec{v}}{|\vec{u}| \cdot
|\vec{v}|}\right)} \]

\highlight{Ist $\vec{u} \cdot \vec{v} = |\vec{u}| \cdot |\vec{v}|$, so ist
$\cos(\phi) = 1$ und somit der Winkel $\phi = 0$. Diese Vektoren nennt
man \emph{kollinear}.}

Beim Skalarprodukt gelten folgende Rechenregeln:
\begin{itemize}
  \item Grösser gleich Null: $\vec{v} \cdot \vec{v} \ge 0$
  \item Nullvektor: Ist $\vec{v} \cdot \vec{v} = 0$, dann ist $\vec{v} = \vec{0}$
  \item Assoziativität ($r \in \mathbb{R}$): $(r \cdot \vec{u}) \cdot \vec{v}
    = r \cdot ( \vec{u} \cdot \vec{v})$
  \item Distributivität: $(\vec{u_1} + \vec{u_2}) \cdot \vec{v}
    = \vec{u_1} \cdot \vec{v} + \vec{u_2} \cdot \vec{v}$
  \item Kommutativität: $\vec{u} \cdot \vec{v} = \vec{v} \cdot \vec{u}$
\end{itemize}

\subsection{Normalenform der Geraden / Ebenen}
Der Normalenvektor $\vec{n}$ steht senkrecht zu einer Gerade $g$. Da das
Skalarprodukt bei zwei senkrecht stehenden Vektoren $0$ ist, kann man
die Gerade zu einem Normalenvektor berechnen:

Ist der Normalenvektor
\[ \vec{n} = \left( \begin{array}{r} -1 \\ 1 \end{array} \right) \]
Dann gibt es aus folgender \emph{Normalenform} eine lineare
Gleichung:
\[ \underbrace{\left( \begin{array}{r} -1 \\ 1 \end{array} \right) \cdot
  \left( \begin{array}{r} x_1 \\ x_2 \end{array} \right)}_{Normalenform}
  = \underbrace{-x_1 + x_2}_{lineare Gleichung} = 0 \]

Somit ist die Gerade $x_1 = x_2$.

Wir haben eine Gerade $g$ mit dem Normalenvektor $\vec{n}$. Die Lage der Gerade
wird mit einem Ortsvektor $\vec{a}$ ("`Aufhänger"') festgelegt. Zudem
haben wir einen Punkt $P$ mit dem Ortsvektor $\vec{r}$. Es gilt nun: $\vec{n}
\cdot (\vec{r} - \vec{a}) = 0$.

\subsubsection{Richtungsvektor einer Gerade}
Der Richtungsvektor zweier Punkte ergibt sich aus deren Differenz:
$$
  \vec{r_p} =
  \begin{pmatrix}
    -1 \\
    -1
  \end{pmatrix},
  \vec{r_q} =
  \begin{pmatrix}
    2 \\
    1
  \end{pmatrix}
$$

$$
  \vec{v} = \vec{r_q} - \vec{r_p} =
  \begin{pmatrix}
    3 \\
    2
  \end{pmatrix}
$$

\subsubsection{Normalenform im $\mathbb{R}^2$}

\subsubsection{Ebene aus 3 Punkten}
$$
A = (1, 2, 4), B = (1, -1, 1), C=(0, 1, 3)
$$
\emph{ARBASCA Formel}

\[
\begin{array}{lcccccl}
& A & R & B-A & S & C-A
\\
E: \vec{x} =
  & \begin{pmatrix}
    1 \\ 2 \\ 4
  \end{pmatrix} +
  & \lambda \cdot
  & \begin{pmatrix}
    1 - 1 \\ -1 - 2 \\ 1 - 4
  \end{pmatrix}  +
  & \mu \cdot
  & \begin{pmatrix}
    0 - 1 \\ 1 - 2 \\ 3 - 4
  \end{pmatrix}
  =
  & \begin{pmatrix}
    1 \\ 2 \\ 4
  \end{pmatrix} +
  \lambda \cdot \begin{pmatrix}
    0 \\ -3 \\ -3
  \end{pmatrix}  +
  \mu \cdot \begin{pmatrix}
    -1 \\ -1 \\ -1
  \end{pmatrix}

\end{array}
\]

\subsubsection{Punktrichtungsform in Normalenform}
\[
E: \vec{x} =
  \begin{pmatrix}
    1 \\ 2 \\ 4
  \end{pmatrix} + \lambda
  \begin{pmatrix}
    0 \\ -3 \\ -3
  \end{pmatrix} + \mu
  \begin{pmatrix}
    -1 \\ -1 \\ -1
  \end{pmatrix}
\]
Kreuzprodukt bestimmen (mit Haus von Niklaus):
\[
  \begin{pmatrix}
    \colorbox{green}{0} \\ \colorbox{green}{-3} \\ -3
  \end{pmatrix} \times
  \begin{pmatrix}
    \colorbox{red}{-1} \\ \colorbox{red}{-1} \\ -1
  \end{pmatrix} =
  \begin{matrix}
    \colorbox{green}{0} \\ \colorbox{green}{-3} \\ -3 \\ \colorbox{green}{0} \\ \colorbox{green}{-3}
  \end{matrix} -
  \begin{matrix}
    \colorbox{red}{-1} \\ \colorbox{red}{-1} \\ -1 \\ \colorbox{red}{-1} \\ \colorbox{red}{-1}
  \end{matrix} =
  \begin{matrix}
    \\
    \\ -3 \cdot (-1) - (-3) \cdot (-1) = 0
    \\ -3 \cdot (-1) - 0 \cdot (-1) = 3
    \\ 0 \cdot (-1) - (-3) \cdot (-1) = -3
  \end{matrix}
\]
Normalenform:
\[
E:
  (\vec{x} -
  \begin{pmatrix}
    1 \\ 2 \\ 4
  \end{pmatrix}) \cdot
  \begin{pmatrix}
    0 \\ 3 \\ -3
  \end{pmatrix} = 0
\]

\subsubsection{Koordinatenform zu Normalenform}
\[
\begin{array}{lll}
    2x_1 + 3x_2 + 4x_3 = 7 & \Rightarrow &
      \begin{pmatrix}
        2 \\ 3 \\ 4
      \end{pmatrix} \cdot \vec{x} - 7 = 0
\end{array}
\]

\subsubsection{Hessesche Normalenform}
Die Hessesche Normalenform erhält man, wenn man die Normalengleichung
$\vec{n} \cdot \vec{r} - b = 0$ durch den Betrag des Normalenvektors
$|\vec{n}|$ teilt. Mit den Definitionen $\vec{n_0} =
\frac{\vec{n}}{|\vec{n}|}$ und $b_0 = \frac{b}{|\vec{n}|}$ ergibt sich
die Form:
\[ \vec{n_0} \cdot \vec{r} - b_0 = 0 \]

\highlight{Berechnung}
$P(-1, 3)$ und $Q(3,0)$
\begin{enumerate}
  \item Normalenvektor bestimmen
  \[
    \begin{array}{llclc}
      \\ \vec{n} \cdot (\vec{r_P} - \vec{r_Q}) & = 0 & \Rightarrow & \vec{n} \cdot \begin{pmatrix} -4 \\ 3 \end{pmatrix} & = 0
      \\
      & & & -4n_x + 3n_y & = 0
      \\
      & & & \vec{n} & = \begin{pmatrix} 3 \\ 4 \end{pmatrix}
    \end{array}
  \]
  \item Parameter b in Normalenform berechnen
  \[
    \begin{array}{llclc}
      \\ \vec{n} \cdot \vec{r_q} - b & = 0 & \Rightarrow & \begin{pmatrix} 3 \\ 4 \end{pmatrix} \cdot \begin{pmatrix} 3 \\ 0 \end{pmatrix} & = b
      \\
      b & = 9 & \Rightarrow & \begin{pmatrix} 3 \\ 4 \end{pmatrix} \cdot \vec{x} - 9 & = 0
    \end{array}
  \]
  \item Betrag Normalenvektor berechnen
  \[
    |\vec{n}| = \sqrt{3^2 + 4^2} = 5
  \]
  \item Hessesche Normalenform
  \[
    \begin{array}{lr}
        \begin{pmatrix}
          \frac{3}{5} \\
          \frac{4}{5}
        \end{pmatrix} \cdot \vec{x} - \frac{9}{5} = 0
      & \Rightarrow
        \begin{pmatrix}
          0.6 \\
          0.8
        \end{pmatrix} \cdot \vec{x} - 1.8 = 0
    \end{array}
  \]
\end{enumerate}
\highlight{Abstände}
Um den Abstand zu einem Punkt zu berechen setzt man den Punkt für $\vec{x}$ in die HNF ein:
\[
Punkt(2, 2) \Rightarrow
\begin{pmatrix}
  0.6 \\
  0.8
\end{pmatrix} \cdot
\begin{pmatrix}
  2 \\
  2
\end{pmatrix}
 - 1.8 = (1.2 + 1.6) - 1.8 = 1
\]

\subsection{Basen und Koordinaten}
Im Erzeugendensystem $\vec{x} = \lambda_1 \cdot \vec{v_1} + \lambda_2
\cdot \vec{v_2} + \dots + \lambda_n \cdot \vec{v_n}$ erzeugen
$\vec{v_1}$ bis $\vec{v_n}$ ein Vektorraum.


%%%%%%%%%%%%%%%%%%%%%%%%%%%%%%%%%%%%%%%%%%%%%%%%%%%%%%%%%%%%%%%%%%%%%%%%
\section{Matrizen}
\subsection{Rechenregeln für Matrizen}
\subsubsection{Addition}
\[ A + B =
  \begin {pmatrix}
    a_{11} & a_{12} & a_{13} & \hdots & a_{1m} \\
    a_{21} & a_{22} & a_{23} & \hdots & a_{2m} \\
    \vdots  & \vdots  & \vdots  &        & \vdots  \\
    a_{n1} & a_{n2} & a_{n3} & \hdots & a_{nm} \\
  \end {pmatrix} \cdot
  \begin {pmatrix}
    b_{11} & b_{12} & b_{13} & \hdots & b_{1m} \\
    b_{21} & b_{22} & b_{23} & \hdots & b_{2m} \\
    \vdots  & \vdots  & \vdots  &        & \vdots  \\
    b_{n1} & b_{n2} & b_{n3} & \hdots & b_{nm} \\
  \end {pmatrix} \]
\[ = \begin {pmatrix}
    a_{11} + b_{11} & a_{12} + b_{12} & a_{13} + b_{13} & \hdots &
      a_{1m} + b_{1m} \\
    a_{21} + b_{21} & a_{22} + b_{22} & a_{23} + b_{23} & \hdots &
      a_{2m} + b_{2m} \\
    \vdots  & \vdots  & \vdots  &        & \vdots  \\
    a_{n1} + b_{n1} & a_{n2} + b_{n2} & a_{n3} + b_{n3} & \hdots &
    a_{nm} + b_{nm} \\
  \end {pmatrix} \]
Beispiel:
\[ A + B =
  \begin {pmatrix} 1 & -1 \\ 2 & 1 \\ \end {pmatrix} +
  \begin {pmatrix} 1 & 0 \\ 0 & 1 \\ \end {pmatrix} =
  \begin {pmatrix} 2 & -1 \\ 2 & 2 \\ \end {pmatrix}
\]

\subsubsection{Multiplikation mit einem Skalar}
Eine Matrix $A$ kann mit einem Skalar $k$ multipliziert werden. Dabei
ist $k \in \mathbb{R}$.
\[ k \cdot A = k \cdot
  \begin {pmatrix}
    a_{11} & a_{12} &  \hdots & a_{1m} \\
    a_{21} & a_{22} &  \hdots & a_{2m} \\
    \vdots  & \vdots   &      & \vdots  \\
    a_{n1} & a_{n2} &  \hdots & a_{nm} \\
  \end {pmatrix} =
  \begin {pmatrix}
    k \cdot a_{11} & k \cdot a_{12} &  \hdots & k \cdot a_{1m} \\
    k \cdot a_{21} & k \cdot a_{22} &  \hdots & k \cdot a_{2m} \\
    \vdots  & \vdots   &      & \vdots  \\
    k \cdot a_{n1} & k \cdot a_{n2} &  \hdots & k \cdot a_{nm} \\
  \end {pmatrix}
\]
Beispiel (für $k \in \mathbb{R}$):
\[ k \cdot A = 5 \cdot
  \begin {pmatrix} 1 & 0 \\ 0 & 1 \\ \end {pmatrix} =
  \begin {pmatrix} 5 & 0 \\ 0 & 5 \\ \end {pmatrix}
\]
\subsubsection{Matrix-Multiplikation}
Wir haben die Matrix $A$ und die Matrix $B$:
\[ A =
  \begin {pmatrix}
    1 & 2 & 4 \\
    3 & 2 & 1 \\
  \end {pmatrix} \text{ und }
  B =
  \begin {pmatrix}
    1 & 1 & 0 & -1 \\
    0 & 3 & 1 & -1 \\
    2 & 0 & 2 & -2 \\
  \end {pmatrix}
\]
Man nimmt die Zeilenvektoren der Matrix $A$ (Das $T$ steht für
Transformiert)
\[ \vec{z_1} = (1, 2, 4)^T ,\vec{z_2} = (3, 2, 1)^T \]
und die Spaltenvektoren der Matrix $B$:
\[ \vec{s_1} \begin {pmatrix} 1 \\ 0 \\ 2 \\ \end {pmatrix},
 \vec{s_2} \begin {pmatrix} 1 \\ 3 \\ 0 \\ \end {pmatrix},
 \vec{s_3} \begin {pmatrix} 0 \\ 1 \\ 2 \\ \end {pmatrix},
 \vec{s_4} \begin {pmatrix} -1 \\ -1 \\ -2 \\ \end {pmatrix}
\]

Das Produkt der Matrix $A$ mit der Matrix $B$ ergibt sich aus den Skalarprodukten
der Zeilen- und Spaltenvektoren (Zeilen der rechten x Spalten der linken Matrix):
\[ A \cdot B =
  \begin {pmatrix}
    \vec{z_1} \cdot \vec{s_1} & \vec{z_1} \cdot \vec{s_2} & \vec{z_1} \cdot \vec{s_3} & \vec{z_1} \cdot \vec{s_4} & \vec{z_1} \cdot \vec{s_2} \\
    \vec{z_2} \cdot \vec{s_1} & \vec{z_2} \cdot \vec{s_2} & \vec{z_2} \cdot \vec{s_3} & \vec{z_2} \cdot \vec{s_4} & \vec{z_2} \cdot \vec{s_2} &
  \end {pmatrix} \]
Das rechnet man am Besten mit einer Tabelle:

\begin{tabular}{|ccc||cccc|}
\hline
        &   &   & 1 & 1 & 0 & -1 \\
  A & $\cdot$  & B & 0 & 3 & 1 & -1 \\
        &   &   & 2 & 0 & 2 & -2 \\
      \hline
      \hline
      1 & 2 & 4 & 9 & 7 & 10 & -11 \\
      3 & 2 & 1 & 5 & 9 & 4 & -7 \\
  \hline
\end{tabular}

Das Produkt hat immer so viele Zelen wie der erste Faktor.
Das heisst jetzt:
\[ A \cdot B =
  \begin {pmatrix}
    1 & 2 & 4 \\
    3 & 2 & 1 \\
  \end {pmatrix} \cdot
  \begin {pmatrix}
    1 & 1 & 0 & -1 \\
    0 & 3 & 1 & -1 \\
    2 & 0 & 2 & -2 \\
  \end {pmatrix} =
  \begin {pmatrix}
    9 & 7 & 10 & -11 \\
    5 & 9 & 4 & -7 \\
\end {pmatrix} \]
Wichtig:
\begin{itemize}
  \item Die Faktoren dürfen nicht vertauscht werden:  $ A \cdot B \ne B \cdot A $
  \item Eine 2x4 und 4x4 Matrix kann auch nicht multipliziert werden.
\end{itemize}

\subsection{Matrizen und ihre Inversen}
Vorgehen: Matrix $A$ links, und eine Einheitsmatrix $E$ rechts hinschreiben. Die
Matrix $A$ zur Einheitsmatrix $E$ umformen ergibt auf der rechten Seite die
Inverse $A^{-1}$. Das ergibt folgende Beziehung:

\[ A \cdot A^{-1} = A^{-1} \cdot A = E \]

Regeln:
\begin{itemize}
  \item Werden in der Matrix $A$ während dem Gauss-Algorithmus
    \emph{Spalten} 1 und 2 vertauscht, muss man in der rechten Matrix
    am Schluss die \emph{Zeilen} 1 und 2 vertauschen.
  \item Nicht jede Matrix ist invertierbar.
\end{itemize}

%%%%%%%%%%%%%%%%%%%%%%%%%%%%%%%%%%%%%%%%%%%%%%%%%%%%%%%%%%%%%%%%%%%%%%%%
\section{Lineare Abbildungen}
\subsection{Koordinaten und Transformation}
Eine lineare Abbildung beschreibt die Abbildung zwischen zwei Vektorräumen
über demselben Körper.

\paragraph{\emph{Spalten der Matrix $=$ die Bilder der Basisvektoren}}
\[ A \cdot \vec{e_1} = \begin {pmatrix}
  0 & -1 & 0 \\
  1 & 0 & 0 \\
  0 & 0 & 1 \\ \end {pmatrix}
\cdot \begin {pmatrix} 1 \\ 0 \\ 0 \\
\end {pmatrix} = \begin {pmatrix} 0 \\ 1 \\ 0 \\ \end {pmatrix}
= \vec{e_1}' = \vec{e_2} \]

\subsection{Determinanten}
\subsubsection{Determinante einer 2x2 Matrix}
Von der Matrix $A$
\[ A =
  \begin {pmatrix} a_{11} & a_{12} \\ a_{21} & a_{22} \\ \end {pmatrix} \]
Ist die Determinante
\[ \text{Det}(A) = a_{11} \cdot a_{22} - a_{21} \cdot a_{12} \]
Produkt der ersten Diagonale minus Produkt der zweiten Diagonale.

\subsubsection{Determinante der invertierten Matrix}
$$ \text{Det}(A^{-1}) = \frac{1}{\text{Det}(A)} $$

\subsubsection{Invertierende einer 2x2 Matrix mit der Determinante berechnen}
\[ A^{-1} = \frac{1}{\text{Det}(A)} \cdot
  \begin {pmatrix}
    a_{22} & -a_{12} \\
    -a_{21} & a_{11} \\
  \end {pmatrix} \]
Kehrwert der Determinalte multipliziert mit der Matrix von A, in welcher
die erste Diagonale vertauscht wurde und die zweite Diagonale mit $(-1)$
multipliziert wurde.

\subsubsection{Regeln für das Rechnen mit Determinanten}
Folgende Regeln sind bei Operationen mit Determinanten zu beachten:
\begin{itemize}
  \item Determinanten könnnen nur bei quadratischen Matritzen berechnet
    werden.
  \item Vertauschen von Zeilen oder Spalten ändert das Vorzeichen der
    Determinante.
  \item Wenn eine Zeile mit $c \ne 0$ multipliziert wird, wird die
    Determinante mit $\frac{1}{c}$ multipliziert.
  \item \highlight{Eine Determinante ist Null, wenn eine gesamte Zeile oder
    eine gesamte Spalte $= 0$ ist.}
  \item Eine Determinante ist Null, wenn zwei Spalten oder zwei Zelen
    gleich sind.
  \item Die Determinante ist Null, wenn Zeilen oder Spalten linear
    abhängig sind.
  \item \highlight{Eine obere Dreiecksmatrix hat als Determinante das Produkt der
    Diagonale.}
  \item $\text{Det}(A \cdot B) = \text{Det}(A) \cdot \text{Det}(B)$
  \item Ist die Matritze $A$ invertierbar, dann ist $\text{Det}(A^{-1}) =
    \frac{1}{\text{Det}(A)}$
  \item \highlight{Ist die {\text{Det}(A) $\neq$ 0}, so ist die Matritze $A$ invertierbar.}
\end{itemize}


\subsubsection{Determinante einer 3x3 Matrix}
Die Determinante einer 3x3 Matrix
\[ A =
  \begin {pmatrix}
    a_{11} & a_{12} & a_{13} \\
    a_{21} & a_{22} & a_{23} \\
    a_{31} & a_{32} & a_{33} \\
  \end {pmatrix} \]
Berechnet sich so:
\[ \text{Det}(A) =
a_{11} \cdot a_{22} \cdot a_{33} +
a_{12} \cdot a_{23} \cdot a_{31} +
a_{13} \cdot a_{21} \cdot a_{32} -
a_{31} \cdot a_{22} \cdot a_{13} -
a_{32} \cdot a_{23} \cdot a_{11} -
a_{33} \cdot a_{21} \cdot a_{12}
\]
Man wiederholt die Zahlen der Matrix hinter der Matrix und multipliziert
alle 'fallenden` Diagonalprodukte und subtrahiert alle 'steigenden`
Diagonalprodukte.

\subsubsection{Determinante einer nxm Matrix}
Matrix reduzieren:
\[ \text{Det}(A) =
  \begin{pmatrix}
    1 & 4 & 3 & 2 \\
    2 & 1 & -1 & 1 \\
    -3 & 2 & 2 & -2 \\
    -1 & -5 & -4 & 1 \\
  \end{pmatrix}
  =
  \begin{pmatrix}
    1 & 4 & 3 & 2 \\
    0 & -7 & -7 & -5 \\
    0 & 14 & 11 & 4 \\
    0 & -1 & -1 & 3 \\
  \end{pmatrix}
  =
  \begin{pmatrix}
    1 & 4 & 3 & 2 \\
    0 & -7 & -7 & -5 \\
    0 & 0 & -3 & -6 \\
    0 & 0 & 0 & \frac{26}{7} \\
  \end{pmatrix}
  = 1 \cdot 78
\]

\subsection{Eigenwerte und Eigenvektoren}

\subsubsection{Eigenwerte}
Indem man die Determinante einer Matrix $ 0 $ setzt, können die Eigegenwerte berechnet werden:

\begin{align*}
  A =
  \begin {pmatrix}
    \colorbox{green}{$3 - \lambda$} & \colorbox{red}{1} \\
    \colorbox{red}{1} & \colorbox{green}{$3 - \lambda$} \\
  \end {pmatrix}
  \\ \Leftrightarrow \colorbox{green}{$(3 - \lambda)$} \cdot \colorbox{green}{$(3 - \lambda)$} - \colorbox{red}{1} \cdot \colorbox{red}{1} = 0
  \\ \Leftrightarrow (\lambda - 2) \cdot (\lambda - 4) = 0
\end{align*}

\subsubsection{Eigenvektor}
$$(A - \lambda \cdot E) \cdot \vec{v} = 0$$

\[
\begin{array}{llcl}
 \lambda_1 = 4: & (A - 4\cdot E) \cdot \vec{v_1} & = & 0
 \\
  &\left( \begin{pmatrix}
     3 & 1 \\
     1 & 3 \\
  \end{pmatrix} - 4 \cdot
  \begin{pmatrix}
   1 & 0 \\
   0 & 1 \\
  \end{pmatrix} \right) \cdot
  \begin{pmatrix}
   x_1 \\
   x_2 \\
  \end{pmatrix} & = & 0
\\
  &\begin{pmatrix}
     -1 & 1 \\
     1 & -1 \\
  \end{pmatrix} \cdot
  \begin{pmatrix}
   x_1 \\
   x_2 \\
  \end{pmatrix} & = & 0
\\
\\ \hline
\\
  &\color{green}{\text{Gauss-Algorithmus}}
  &\begin{array}{rr|r}
  -1x_1 & 1x_2 & 0 \\
  1x_1 & -1x_2 & 0 \\
  \end {array} & \color{green}{\text{Wähle }x_2=1}
\\
  &\Rightarrow
  x_1 & = & 1
\\
  &\Rightarrow
  x_2 & = & 1
\\
  &\Rightarrow
   \vec{v_1} & = & \begin{pmatrix} 1 \\ 1 \end{pmatrix}
\end{array}
\]

\[
\begin{array}{llcl}
 \lambda_1 = 2: & (A - 2\cdot E) \cdot \vec{v_2} & = & 0
 \\
  & \left( \begin{pmatrix}
     3 & 1 \\
     1 & 3 \\
  \end{pmatrix} - 2 \cdot
  \begin{pmatrix}
   1 & 0 \\
   0 & 1 \\
  \end{pmatrix} \right) \cdot
  \begin{pmatrix}
   x_1 \\
   x_2 \\
  \end{pmatrix} & = & 0
\\
  &\begin{pmatrix}
     1 & 1 \\
     1 & 1 \\
  \end{pmatrix} \cdot
  \begin{pmatrix}
   x_1 \\
   x_2 \\
  \end{pmatrix} & = & 0
\\
\\ \hline
\\
  &\color{green}{\text{Gauss-Algorithmus}}
  &\begin{array}{rr|r}
  1x_1 & 1x_2 & 0 \\
  1x_1 & 1x_2 & 0 \\
  \end {array} & \color{green}{\text{Wähle }x_2=1}
\\
  &\Rightarrow
  x_1 & = &  -1
\\
  &\Rightarrow
   x_2 & = & 1
\\
  &\Rightarrow
   \vec{v_2} & = & \begin{pmatrix} -1 \\ 1 \end{pmatrix}
\end{array}
\]
\subsubsection{Kanonische Basis zu Eigenvektoren}
\[
\begin{pmatrix}
  1 & 2 \\
  1 & 1
\end{pmatrix} = B
\]

\subsubsection{Berechnung der Eigenvektoren}
\emph{Als Beispiel mit einer 2x2 Matrix}

\textbf{Eigenwerte} = Nullstellen des charakteristischen Polynoms
\begin{enumerate}
\item Charakteristisches Polynom durch die Determinante mit der Einheitsmatrix \[ \text{CharPol}(A, \lambda) = \text{Det}(A - \lambda \cdot E) \]
\item Determinante nach 0 auflösen ergeben die Eigenwerte: \[ \lambda_1 = x, \lambda_2 = y \]
\end{enumerate}

\textbf{Eigenvektoren} = Lösungen des linearen Gleichungssystems
\begin{enumerate}
\item Eigenwerte $\lambda_1$ bzw. $\lambda_2$ einsetzen und Gleichungssystem lösen
\[ \lambda_1 = x: (A - \lambda \cdot E)\cdot \vec{v} = 0
\text{ bzw. }
\lambda_2 = y: (A - \lambda \cdot E)\cdot \vec{v} = 0 \]

\item Daraus ergeben sich die Einheitsvektoren \[ EV_{\lambda_1 = x} = \begin{pmatrix} v_1 \\ v_2 \end{pmatrix} \text{ bzw. } EV_{\lambda_2 = y} = \begin{pmatrix} v_1 \\ v_2 \end{pmatrix} \]
\end{enumerate}

\textbf{Eigenraum}
Der Eigenvektor kann ein beliebiges Vielfaches sein. Daraus folgt der Eigenraum:
\[ Eig_{\lambda_1 = x} = \left \{
    \vec{x} \in \mathbb{R}^2 \middle |
      \vec{x} = t \cdot \left( \begin {array} {c} v_1 \\ v_2 \end {array} \right),
      t \in \mathbb{R}
\right \} \]
\begin{center} bzw. \end{center}
\[ Eig_{\lambda_2 = y} = \left \{
    \vec{x} \in \mathbb{R}^2 \middle |
      \vec{x} = t \cdot \left( \begin {array} {c} v_1 \\ v_2 \end {array} \right),
      t \in \mathbb{R}
\right \} \]

%%%%%%%%%%%%%%%%%%%%%%%%%%%%%%%%%%%%%%%%%%%%%%%%%%%%%%%%%%%%%%%%%%%%%%%%
\section{Begriffe}
\subsection{Ist Teiler von}
Folgendes heisst $a$ ist ein Teiler von $b$:
\[ a | b\]
\subsection{Summenformel}
\[\sum \limits_{k=1}^n k = 1 + 2 + ... + n \]
Man kann sich das wie eine For-Schleife vorstellen:
\begin{verbatim}
for i in `seq k n`
do
  SUM=$(($SUM + i))
done
\end{verbatim}
\subsection{Produkteformel}
\[\prod \limits_{k=1}^{n}k = 1\cdot 2\cdot ... \cdot n \]

%%%%%%%%%%%%%%%%%%%%%%%%%%%%%%%%%%%%%%%%%%%%%%%%%%%%%%%%%%%%%%%%%%%%%%%%
% Inhalt Ende
%%%%%%%%%%%%%%%%%%%%%%%%%%%%%%%%%%%%%%%%%%%%%%%%%%%%%%%%%%%%%%%%%%%%%%%%
\end{document}

% EOF
