%%%%%%%%%%%%%%%%%%%%%%%%%%%%%%%%%%%%%%%%%%%%%%%%%%%%%%%%%%%%%%%%%%%%%%%%
%
% header.tex
%
%%%%%%%%%%%%%%%%%%%%%%%%%%%%%%%%%%%%%%%%%%%%%%%%%%%%%%%%%%%%%%%%%%%%%%%%

%%%%%%%%%%%%%%%%%%%%%%%%%%%%%%%%%%%%%%%%%%%%%%%%%%%%%%%%%%%%%%%%%%%%%%%%
% Dokument
\documentclass[12pt,oneside,a4paper,parskip]{scrartcl}
\usepackage[inner=2.5cm,outer=2.5cm,top=2.5cm,bottom=2.5cm,includeheadfoot]{geometry}  % Seitenrand

% Palatino Schriftart
\usepackage[sc]{mathpazo}    % Palatino Schriftart
\renewcommand\sfdefault{ppl}

% Sprache, Kodierung
\usepackage{ngerman}
\usepackage[T1]{fontenc}
\usepackage[utf8]{inputenc}

% Weitere Pakete
\usepackage{graphicx}
\usepackage{color}
\usepackage[table]{xcolor} % http://ctan.org/pkg/xcolor
\usepackage{fancybox}
\usepackage[colorlinks=true, linkcolor=blue, urlcolor=blue, citecolor=blue]{hyperref}
\usepackage{amsmath}
\usepackage{amssymb}
\usepackage{comment}

% Jede Überschrift 1 auf neuer Seite
\let\stdsection\section
\renewcommand\section{\newpage\stdsection}

%%%%%%%%%%%%%%%%%%%%%%%%%%%%%%%%%%%%%%%%%%%%%%%%%%%%%%%%%%%%%%%%%%%%%%%%
% Angaben zum Dokument
\subject{\SUBJECT}
\title{\TITLE}
\author{\AUTHOR \\ \EMAIL}

%%%%%%%%%%%%%%%%%%%%%%%%%%%%%%%%%%%%%%%%%%%%%%%%%%%%%%%%%%%%%%%%%%%%%%%%
% PDF-Metainformationen
\hypersetup{
  pdftitle    = {\TITLE},
  pdfsubject  = {\SUBJECT},
  pdfauthor   = {\AUTHOR, \EMAIL},
  pdfkeywords = {\KEYWORDS} ,
  pdfcreator  = {pdflatex},
  pdfproducer = {LaTeX with hyperref}
}

%%%%%%%%%%%%%%%%%%%%%%%%%%%%%%%%%%%%%%%%%%%%%%%%%%%%%%%%%%%%%%%%%%%%%%%%
% Kopf- und Fusszeile
\usepackage{fancyhdr}
\pagestyle{fancy}
\fancyhf{}
\renewcommand{\headrulewidth}{0.5pt}
\fancyhead[L]{\nouppercase{\TITLE}}
\fancyhead[R]{\nouppercase{\leftmark}}
\renewcommand{\footrulewidth}{0.5pt}
\fancyfoot[L]{\AUTHOR}
\fancyfoot[R]{\thepage}

%%%%%%%%%%%%%%%%%%%%%%%%%%%%%%%%%%%%%%%%%%%%%%%%%%%%%%%%%%%%%%%%%%%%%%%%
% Dokument start
\begin{document}

% Titelseite
\begin{titlepage}
  \maketitle
  \thispagestyle{empty} % Don't start page numbers on this page
  \vfill
  \begin{figure}[!htbp]
    \centering
    \includegraphics[width=300px]{hsr_logo.pdf}
  \end{figure}
\end{titlepage}

% Inhaltsverzeichnis
\setcounter{page}{1}
\tableofcontents

% CC BY-SA Lizenz
\small{
\vfill
Diese Zusammenfassung basiert auf der Vorlesung und auf dem Skript von \emph{\TITLE} der
HSR vom Herbstsemester 2012. \\ \\
\begin{center}
\includegraphics[width=130px]{cc.pdf}
\end{center}
CC BY-SA by Emanuel Duss
(\href{mailto:emanuel.duss@gmail.com}{\tt emanuel.duss@gmail.com})
  % \href{http://emanuelduss.ch}{\tt http://emanuelduss.ch}
% \begin{center}
% 
% 
% Dieses Dokument steht unter der \\
% \emph{Creative Commons \\ Namensnennung - Weitergabe unter gleichen Bedingungen 3.0 Schweiz Lizenz} \\
% 
 % \begin{figure}[!htbp]
   % \centering
 % \end{figure}
% 
% \href{http://creativecommons.org/licenses/by-sa/3.0/ch/}{\tt http://creativecommons.org/licenses/by-sa/3.0/ch/}
% \end{center}
}
\newpage

% EOF
